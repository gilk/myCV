%&latex
%% Derived from: `cvctan.tex'

\documentclass[a4paper]{article}

\usepackage{tabularx}
\usepackage[left=3cm,right=3cm,top=2.7cm,bottom=3cm,nohead,nofoot]{geometry}
\usepackage{palatino, url, multicol}

%\usepackage{doublespace}
%\setstretch{1.2}

%\usepackage{ae}
\usepackage[T1]{fontenc}
\usepackage{CV}
\usepackage[normalem]{ulem}
%\usepackage[none]{hyphenat}%%%%

\begin{document}

\pagestyle{empty}

%Ueberschrift
\begin{center}
\LARGE{\textsc{Gil Menachem Kogan}}
\end{center}
%\vspace{1.5\baselineskip}
\begin{multicols}{2}
\section{Address}
\begin{flushleft}
  62 Brook Road \\
  Newbury Park \\
  Essex  \\
  IG2 7EY \\

\end{flushleft}
\section{Contact Details}
\begin{flushleft}
  
  Phone: (+44) 78 515 43733 \\
  Email: gilkogan@googlemail.com
  \newline 
  \newline 
\end{flushleft}
\end{multicols}

\section{Work Experience}

\begin{CV}

\item[06/2014--Present] \textbf{Data Scientist, \textsc{CrowdEmotion}} As the sole data scientist I am responsible for the entire analysis chain of data. From meeting with clients to determine requirements, to ensuring those requirements are met; including making the tools needed to do so. For example, this has involved writing an API client and using it to pull and curate relevant data for individual clients. Then aggregate the data in a way that makes sense to the client to empower them to make business decisions. I have developed new ways of aggregating and describing complicated data in simple ways to facilitate fast and consistent analysis. Also some portion of time is spent in the technical team doing development work.

\item[10/2008--03/2013] \textbf{Researcher in high energy physics at Imperial College London}\\
Analysis of the data selection criteria and data quality of events at the Super-Kamiokande detector in the Tokai to Kamioka (T2K) experiment, Japan. I was based at J-PARC in Tokai-mura, Japan for two years. My analysis involved tuning the event selection criteria on a data set of several million events. This was done by creating and developing better algorithms in C++ and Python scripts. I also gained some experience with the GRID at CERN where some of T2K's data is stored. My on-site duties included work testing, installing and commissioning of detector modules for the ND280 detector. I was the first student to become a detector expert, a specialised role which put me in charge of a multi--million pound detector.

\item[10/2011--02/2012] \textbf{Computing lab demonstrator, Imperial College, London.} Instructing undergraduates in programming in C++ and helped guide them through simple computing projects, troubleshooting and advising them on good programming practice.

\item[07/2007--08/2007] \textbf{UROP (Undergraduate Research Opportunity Programme)
placement,
Plasma Physics Group at Imperial College, London.} Working on making computer
models of high density plasmas to determine the light propagation properties
of such systems with the aim of improving models of reactions occurring in the
Sun.
% 
% \item[09/2006--12/2006] Bartender, The Holland Club, London. I worked as
% part of a team to organise stock and serve customers.
% 
% \item[03/2006--04/2006] Stall Manager, Camden Market, London. I managed %a
% stall in Camden market, where my responsibilities included food preparation
%and customer service whilst also managing staff.

\item[06/2004--07/2004] \textbf{Lab Assistant, Weizmann Institute.} Participation
in the Dr. Bessie Lawrence International Summer Science Institute working
on gesture recognition in the department of theoretical mathematics.

%\item[10/2003--05/2004] \textbf{Voluntary teachers assistant, Ilford Jewish Primary
%School, Essex.} Providing extra tuition for underachieving pupils.
%
%\item[10/2002--05/2003] \textbf{Voluntary teachers assistant, Gilbert Colvin Primary School, Essex.} Giving extra curricular mathematics tuition for extension
%paper SAT candidates.

%\item[07/2003--08/2003] \textbf{Finalist, Schools Aerospace Challenge.} A short introductory
%course in aeronautical engineering after being one of the top eighteen teams
%in the country to design modifications for the JSF.

% \item[06/2001--08/2001] Food vendor in Camden market, London. Preparing %food
% and dealing with customers.
% 
% \item[10/1998--02/2000] Sales assistant at Gallia Textiles, London. Working %in a habdashary dealing with customers and organising stock.



\end{CV}

\pagebreak
\section{Skills}
\begin{CV}
\item[\textbf{Computing}] Have experience programming in several languages including; C++, Matlab, Fortran, Python and Machine/Assembly Code. Have
experience with Windows/OS X/linux/unix, mySQL, git, R, racket. Proficient in Microsoft Office, iWork and
\LaTeX. 
\item[\textbf{Practical}] Over the course of my career I have worked in many different environments and roles. I am extremely comfortable in both labs and offices. I have worked on projects individually and as part of a team, as well as taking leading roles when needed. I have experience giving presentations and writing explaining technical material to audiences of all levels.


%Due to my undergraduate studies in physics I am capable
%of working in both laboratory and office environments, working individually
%or incorporated as part of a team. I have also gained experience in giving presentations
%and communicating ideas.

\item[\textbf{Languages}] Native English speaker as well as conversational Russian, Hebrew, basic Japanese and GCSE level French.
\end{CV}


%\pagebreak
\section{Education}

\begin{CV}



%Project title : \emph{Determining neutral current quasi elastic backgrounds to a muon neutrino disappearance measurement at the Super--Kaimokande detector using Super--Kamiokande and ND280 data}, Supervisor: Dr. D. Wark

%Working as part of the T2K collaboration, the world's leading neutrino oscillation experiment. I have been heavily involved in the construction and commission of the near detector in Tokai--mura, Japan. I have also been developing analysis methods at Super--Kamiokande implemented in C++ and FORTRAN. My analysis primarily concerns studying the quality and validity of data through statistical modelling of physical processes.

\item[10/2004--06/2008] \textbf{Upper second class MSci in Physics at Imperial College London}\\
An extensive course on a wide range of areas in physics. Advanced final year options taken include; advanced particle physics, quantum field theory and optical communication physics. I also took options in computational physics, mathematical analysis and quantum optics.
My final year project was to build a toy Monte Carlo of electroweak penguin decays at LHCb detector in CERN implemented in C++. During the project I was fully embedded with the LHCb research group at Imperial.
The course also included several transferable skills modules that focus on team work and presentation skills.
%Project title: \emph{Search for deviations from the Standard Model with the LHCb detector.}; Looking at sensitivity to new physics by building a toy Monte Carlo of electroweak penguin decays at the LHCb detector at CERN implemented in C++.
%Supervisor: Dr. U. Egede

\item[09/1997--06/2004] \textbf{Ilford County High School, Essex.}\\
Three A-levels Grade A in Physics, Maths and Electronics. Nine GCSE's and
one GCSE/SC grades A-C.
Also won the A-Level Physics Prize.


\end{CV}
\section{Interests} % (fold)
\label{sec:interests}

\begin{CV}
	
\item[Technology] I have an avid interest in cutting edge science and technology. I support this interest primarily by following online communities and reading journal repositories.

\item[Computing] I enjoy learning new programming languages and like to undertake small projects to do so. I taught myself Python by working through problems on projecteuler.net. I recently wrote a script to query Google Maps to determine how well a city is laid out by comparing `metropolitan' (walking) distance against straight line distance of two randomly thrown points.

\item[Data] I follow and attend events like LonData to learn more of the commercial side of the field supplementing the practical knowledge I have gained during my career.

\item[Girlguiding] For the past year I have been volunteering within the Kensington and Chelsea division. I have earned several badges in the process.

\end{CV}

% section interests (end)


\section{References}

\noindent References available on request.

%
%\noindent These persons are familiar with my professional qualifications and my character:
%
%\begin{table}[h]
%\begin{tabular}{@{}lll@{}}
%\textbf{Dr. U. Egede} \ \ \ \ \ \\
%MSci supervisor & Phone: & +44 (0)20 7594 7688 \\
%& Email: & u.egede@imperial.ac.uk \\
%Imperial College, London
%\end{tabular}
%\end{table}
%
%\begin{table}[h]
%\begin{tabular}{@{}lll@{}}
%\textbf{Prof. S. Lebedev}\\
%Personal Tutor & Phone: & +44 (0)20 7594 7748  \\
%& Email: & s.lebedev@imperial.ac.uk \\
%Imperial College, London \\
%\end{tabular}
%\end{table}

%\begin{table}[h]
%\begin{tabular}{@{}lll@{}}
%\textbf{Tim Durkin} \ \ \ \ \ \\
%Electronics engineer & Phone: & +44 (0)20 7594 7748  \\
%& Email: & tim.durkin@stfc.ac.uk \\
%Rutherford Appleton Laboratory
%\end{tabular}
%\end{table}

%\begin{table}[h]
%\begin{tabular}{@{}lll@{}}
%\textbf{Dr. M. Murdoch} \ \ \ \ \ \\
%Job title &
%Email: & mmurdoch@hep.ph.liv.ac.uk \\
%University of Liverpool
%\end{tabular}
%\end{table}


%\vspace{3pt}
%\noindent \textbf{Tim Durkin}\\
%Electronics engineer\\
%STFC Rutherford Appleton Laboratory (PPD)\\
%Email: tim.durkin@stfc.ac.uk\\
%
%\noindent \textbf{Dr Matthew Murdoch}\\
%Post--Doctoral Research Associate\\
%University of Liverpool, Particle Physics Group\\
%Email: mmurdoch@hep.ph.liv.ac.uk\\

\end{document}

%Tabellen
\begin{table}[htbp] \centering%
\begin{tabular}{lll}\hline\hline
1 & 2 & 3 \\ \hline
1 & \multicolumn{2}{c}{2} \\
\hline
\end{tabular}
\caption{Titel\label{Tabelle: Label}}
\end{table}







